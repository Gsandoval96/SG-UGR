\documentclass[11pt,a4paper]{article}
\usepackage[utf8]{inputenc}
\usepackage[spanish]{babel}	%Idioma
\usepackage{amsmath}
\usepackage{url}
\makeatletter
\makeatother
\usepackage{amsfonts}
\usepackage{amssymb}
\usepackage{graphicx} 	%Añadir imágenes
\usepackage{geometry}	%Ajustar márgenes
\usepackage[export]{adjustbox}[2011/08/13]
\usepackage{float}
\restylefloat{table}
\usepackage[hidelinks]{hyperref}
\usepackage{titling}
\graphicspath{}
\usepackage{multirow}
\usepackage{caption}
\usepackage{multicol}
\usepackage{array}
\usepackage{eurosym}


%Opciones de encabezado y pie de página:
\usepackage{fancyhdr}
\pagestyle{fancy}
\lhead{Grado en Ingeniería Informática}
\rhead{Diseño de la Aplicación}
\lfoot{Sistemas Gráficos}
\cfoot{}
\rfoot{\thepage}
\renewcommand{\headrulewidth}{0.4pt}
\renewcommand{\footrulewidth}{0.4pt}

%Opciones de fuente:
\usepackage[utf8]{inputenc}
\usepackage[default]{sourcesanspro}
\usepackage{sourcecodepro}
\usepackage[T1]{fontenc}

\setlength{\parindent}{15pt}
\setlength{\headheight}{15pt}
\setlength{\voffset}{10mm}

% Custom colors
\usepackage{color}
\definecolor{deepblue}{rgb}{0,0,0.5}
\definecolor{deepred}{rgb}{0.6,0,0}
\definecolor{deepgreen}{rgb}{0,0.5,0}
\hypersetup{
    colorlinks=true,
    linkcolor=black,
    urlcolor=blue,
}

\usepackage{listings}

\begin{document}
\sloppy
\begin{titlepage}
  \centering
  \includegraphics[width=0.7\textwidth]{logo.png}\par\vspace{1cm}
  {\scshape\large Sistemas Gráficos \par} \vspace{1cm}
  {\huge\bfseries Diseño e implementación de \\ un sistema gráfico \par}
  \vspace{0.4cm}
  {\large\bfseries ---Diseño---\\}
  \vspace{0.6cm}
  {\large\itshape  Guillermo Sandoval Schmidt  \par} \vspace{1.00cm}
  Curso 2019-2020 \\
  \vfill

  % Bottom of the page
  {\large 12 de junio de 2020 \par}
\end{titlepage}

\pagenumbering{gobble}
\pagenumbering{arabic}
\tableofcontents
\thispagestyle{empty}

\newpage

\section{Grupo de prácticas}
\begin{itemize}
    \item \textbf{Nombre de Alumno:} Guillermo Sandoval Schmidt
    \item \textbf{Nombre de la Aplicación:} Tetris
\end{itemize}

\section{Introducción}

\section{Diseño}

\subsection{Diagrama de clases}
\subsection{Clases}

\section{Físicas}

\subsection{Movimientos}
\subsubsection{Colisiones}

\section{Animaciones}

\subsection{Apartado de jugabilidad}
\subsection{Apartado estético}

\section{Otros elementos}

\subsection{Materiales}
\subsection{Luces}
\subsection{Música}
\subsection{Cámara}
\subsection{Controles}

\section{Demo}

\section{Bibliografía}

\end{document}
